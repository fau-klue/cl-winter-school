% \documentclass[handout,t]{beamer} % HANDOUT
% \documentclass[handout,notes=show,t]{beamer} % NOTES
\documentclass[aspectratio=1610,t]{beamer} % SLIDES
\usetheme{ccl}

\usepackage[utf8]{inputenc}
\usepackage[ngerman]{babel}
\usepackage{hyperref}
\usepackage{bibentry}
\usepackage{pifont}
\usepackage{array,hhline,booktabs}   % modern tabular displays
\usepackage{marvosym}
\usepackage{kotex}
\usepackage{nicefrac}


%----------------------------------%
% Abkürzende mathematische Befehle %
%----------------------------------%
\newcommand{\R}{\mathbb{R}}			% reelle Zahlen
\newcommand{\A}{\mathcal{A}}                    % Ereignisraum
\newcommand{\Cov}{\operatorname{Cov}}		% Kovarianz
\newcommand{\Var}{\operatorname{Var}}		% Varianz
\newcommand{\Prob}{\mathbb{P}}			% Wahrscheinlichkeitsmaß
\newcommand{\WRaum}{(\Omega,\A,\Prob)}		% Wahrscheinlichkeitsraum
\newcommand{\ind}{\mathbbmss{1}} 		% Indikator-Eins



%%%%%%%%%%%%%%%%%%%%%%%%%%%%%%%%%%%%%%%%%%%%%%%%%%%%%%%%%%%%%%%%%%%%%%
%% Titlepage

\title{\textbf{Keywords und Kollokate}}
\subtitle{Wörter, Texte \& Frequenzen\\Wintersemester 2021/22}
\author[Blombach \& Heinrich]{Andreas Blombach \& Philipp Heinrich}
\institute[]{%\url{philipp.heinrich@fau.de}
  Lehrstuhl für Korpus- und Computerlinguistik\\
  Friedrich-Alexander-Universität Erlangen-Nürnberg\\
  % {\secondary{\url{philipp.heinrich@fau.de}}}}
  }
\date[24.01.2022]{Erlangen, 24.01.2022}
\setbeamertemplate{navigation symbols}{} %remove navigation symbols

\begin{document}
\frame{\titlepage}
\hideLogo

\section{Keywords}

\begin{frame}
  \frametitle{Keywords}
  \begin{itemize}
  \item \emph{Keywords} sind Wörter, die in einem gegebenen Korpus überdurchschnittlich oft vorkommen -- im Vgl. zur Häufigkeit in einem \emph{Referenzkorpus}
  \item[]
  \item \emph{Assoziationsmaße} quantifizieren den Vergleich mittels einer einzelnen reellen Zahl, basierend auf Intuition und/oder statistischen Verfahren
  \item[]
  \item \emph{Keyness} ist ein textuelles, kein sprachliches Feature
    \begin{itemize}
    \item gesprochene Sprache vs. geschriebene Sprache
    \item soziale Medien vs. Zeitungen
    \item Hochliteratur vs. Groschenromane
    \item links-liberale Zeitungen vs. rechts-konservative Zeitungen
    \item Grüne vs. AfD
    \item \ldots
    \end{itemize}
  \item[]
  \item Anwendung bspw. in der \emph{Diskursanalyse}, \emph{Indexerstellung}, \ldots
  \end{itemize}
\end{frame}

\begin{frame}[c]
  \frametitle{Keywords in CQPweb}
  \centering
  \includegraphics[width=0.7\textwidth]{../img/keywords_cqpweb}
\end{frame}

\section{Kollokate}
\begin{frame}
  \frametitle{Kollokate}
  \begin{itemize}
  \item \emph{Kollokate} eines Wortes (dem \emph{Knoten}, oder engl. \emph{node}) sind Wörter, die häufig in dessen Umgebung auftreten (Ko-Okkurrenz)
  \item[]
  \item Einblick in die Semantik des Wortes, vgl. Firth's (1957) \emph{distributional hypothesis}:
    \begin{center}
      \emph{You shall know a word by the company it keeps!}
    \end{center}
  \item[]
  \item hier: Kollokation als Phänomen, das in Korpora empirisch beobachtbar ist
    \begin{itemize}
    \item Kollokate von \glqq Atomkraft\grqq\ im GermaParl
    \item Kollokate von \glqq Impfung\grqq\ auf Twitter
    \end{itemize}
  \item[]
  \item Anwendung bspw. in der \emph{Diskursanalyse}, \emph{Lexikographie}, \ldots
  \item[]
  \item Kollokate $\neq$ Mehrworteinheiten, Idiome, Phraseologismen, \ldots
  \end{itemize}
\end{frame}



\begin{frame}
  \frametitle{Kollokate von \emph{bucket} (noun)}
  \vspace{-.5cm}
  \begin{center}
    \begin{scriptsize}
      \begin{tabular}{>{\itshape}lr}
        \toprule
        \textbf{noun} & $f$ \\
        \midrule
      water & 183 \\
      spade &  31 \\
    plastic &  36 \\
       slop &  14 \\
       size &  41 \\
        mop &  16 \\
     record &  38 \\
     bucket &  18 \\
        ice &  22 \\
       seat &  20 \\
       coal &  16 \\
    density &  11 \\
    brigade &  10 \\
  algorithm &   9 \\
     shovel &   7 \\
  container &  10 \\
       oats &   7 \\
       sand &  12 \\
      Rhino &   7 \\
  champagne &  10 \\
        \bottomrule
      \end{tabular}
      \hspace{5mm}
      \begin{tabular}{>{\itshape}lr}
        \toprule
        \textbf{verb} & $f$ \\
        \midrule
      throw & 36 \\
       fill & 29 \\
  randomize &  9 \\
      empty & 14 \\
        tip & 10 \\
       kick & 12 \\
       hold & 31 \\
      carry & 26 \\
        put & 36 \\
      chuck &  7 \\
       weep &  7 \\
       pour &  9 \\
      douse &  4 \\
      fetch &  7 \\
      store &  7 \\
       drop &  9 \\
       pick & 11 \\
        use & 31 \\
       tire &  3 \\
      rinse &  3 \\
        \bottomrule
      \end{tabular}
      \hspace{5mm}
      \begin{tabular}{>{\itshape}lr}
        \toprule
        \textbf{adjective} & $f$ \\
        \midrule
          large & 37 \\
  single-record &  5 \\
           cold & 13 \\
     galvanized &  4 \\
     ten-record &  3 \\
           full & 20 \\
          empty &  9 \\
       steaming &  4 \\
     full-track &  2 \\
   multi-record &  2 \\
          small & 21 \\
          leaky &  3 \\
     bottomless &  3 \\
     galvanised &  3 \\
           iced &  3 \\
          clean &  7 \\
         wooden &  6 \\
            old & 19 \\
       ice-cold &  2 \\
     anti-sweat &  1 \\
        \bottomrule
      \end{tabular}
    \end{scriptsize}
  \end{center}
\end{frame}

\begin{frame}[c]
  \frametitle{fensterbasierte Kollokate in CQPweb}
  \begin{center}
    \includegraphics[width=\textwidth]{../img/collocations_cqpweb-2}
  \end{center}
\end{frame}

\subsection{Kookkurrenzarten}
\begin{frame}
  \frametitle{textuelle Kookkurrenz}
  \framesubtitle{textual cooccurrence / segment-based cooccurrence}
  \begin{center}
    Text = Artikel, Paragraph, Tweet, Post, Satz, \ldots

    \vspace{.5cm}
    \colorbox{blue!20!white}{%
      \includegraphics[width=.8\textwidth]{../img/cooc_sentence}}
  \end{center}
\end{frame}

\begin{frame}[c]
  \frametitle{textuelle Kookkurrenz (Satzfenster)}

  \centering
  \includegraphics[width=\textwidth]{../img/cont_table_sentence}

\end{frame}

\begin{frame}
  \frametitle{Oberflächenkookkurrenz}
  \framesubtitle{surface cooccurrence / distance-based cooccurrence}
  \begin{center}
    fensterbasiert, abgeschnitten an entsprechenden Grenzen

    \vspace{.5cm}
    \colorbox{blue!20!white}{%
      \includegraphics[width=\textwidth]{../img/cooc_distance}}
  \end{center}
\end{frame}

\begin{frame}[c]
  \frametitle{Oberflächenkookkurrenz (L4, R4)}

  \centering
  \includegraphics[width=\textwidth]{../img/cont_table_distance}

\end{frame}

\begin{frame}
  \frametitle{syntaktische Kookkurrenz}
  \framesubtitle{syntactic cooccurrence / relational cooccurrence}
  \begin{center}
    Ausnutzung syntaktischer Strukturen

    \vspace{.5cm}
    \colorbox{blue!20!white}{%
      \includegraphics[width=\textwidth]{../img/cooc_syntactic}}
  \end{center}
\end{frame}

\begin{frame}[c]
  \frametitle{syntaktische Kookkurrenz}

  \centering
  \includegraphics[width=\textwidth]{../img/cont_table_syntactic}

\end{frame}

\section{Assoziationsmaße}

\begin{frame}
  \frametitle{Kontingenz und Indifferenz}
  \begin{itemize}
  \item Kontingenztabelle (beobachtete Häufigkeiten):
    \bgroup
    \def\arraystretch{1.5}
    \[
      \begin{array}{c|c|c|c}
        & \text{word} & \text{other words} & \\
        \hline
        \text{corpus}_1 & O:=O_{11} & O_{12} &= R_1\\
        \text{corpus}_2 & O_{21} & O_{22} & = R_2\\
        \hline
        & =C_1 & =C_2 & =N
      \end{array}
    \]
    \egroup
  \item[]
  \item Indifferenztabelle (erwartete Häufigkeiten \emph{bei Unabhängigkeit}):
  \bgroup
    \def\arraystretch{1.5}
    \begin{center}
      \begin{tabular}{c|c|c|c}
        & \text{word} & \text{other words} & \\
        \hline
        $\text{corpus}_1$ & $E:=E_{11}=\frac{R_1C_1}{N}$ & $E_{12}=\frac{R_1C_2}{N}$ & $=R_1$\\
        $\text{corpus}_2$ & $E_{21}=\frac{R_2C_1}{N}$ & $E_{22}=\frac{R_2C_2}{N}$ & $=R_2$\\
        \hline
        & $=C_1$ & $=C_2$ & $=N$
      \end{tabular}
    \end{center}
    \egroup
  \end{itemize}
\end{frame}

\begin{frame}
  \frametitle{Assoziationsmaße}
  Quantifikation der Abweichung:
  \def\arraystretch{1.5}
    \[
      \begin{array}{r|c|c|c}
        & \text{word} & \text{other words} &\\
        \hline
        \text{corpus}_1 & O \text{ vs. }E& O_{12} \text{ vs. }E_{12} &=R_1\\
        \text{corpus}_2 & O_{21} \text{ vs. }E_{21} & O_{22} \text{ vs. }E_{22} &=R_2\\
        \hline
        & =C_1 & =C_2 &=N
      \end{array}
    \]

  \bigskip
  \begin{columns}
    \begin{column}{0.4\textwidth}
      \begin{itemize}
      \item $\text{log-ratio}=\log\frac{\nicefrac{O_{11}}{R_1}}{\nicefrac{O_{21}}{R_2}}$
      \item $\text{MI} = \log_2\nicefrac{O}{E}$
      \item $\text{t-score} = \frac{O - E}{\sqrt{O}}$
      \end{itemize}
    \end{column}
    \begin{column}{0.6\textwidth}
      \begin{itemize}
      \item $LL = 2\sum_{ij} O_{ij}\log\frac{O_{ij}}{E_{ij}}$
      \item $\chi^2 = \sum_{ij} \frac{(O_{ij} - E_{ij})^2}{E_{ij}}$
      \item \ldots
      \end{itemize}
    \end{column}
  \end{columns}
\end{frame}

\section{Software}
\begin{frame}
  \frametitle{Software}
  \begin{itemize}
  \item Berechnung der Assoziationsmaße unkompliziert
    \begin{itemize}
    \item \texttt{R}: einfache Datensatzmanipulation
    \item Python: \href{https://pypi.org/project/association-measures/}{association-measures}
    \item CLI (Perl): \href{http://www.collocations.de/software.html}{UCS toolkit}
    \end{itemize}
  \item[]
  \item korrektes und effizientes Zählen am besten nach Korpusindexierung in CWB
    \begin{itemize}
    \item \texttt{R}: \href{https://cran.r-project.org/web/packages/polmineR/index.html}{PolmineR}
    \item Python: \href{https://pypi.org/project/cwb-ccc/}{cwb-ccc}
    \item GUI (PHP): \href{https://cwb.sourceforge.io/cqpweb.php}{CQPweb}
    \end{itemize}
  \end{itemize}
\end{frame}


\end{document}
%%% Local Variables:
%%% mode: latex
%%% TeX-master: t
%%% End:
